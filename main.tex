%import template
\input{./template.tex}

% DocInfo
\newcommand{\SUBJECT}{}
\newcommand{\TITLE}{Cheat Sheet Objektorientierte Programmierung}

\begin{document}

%import front page
% \input{./front.tex}

%do multicols
\begin{multicols*}{5}
    \setlength{\columnseprule}{0.4pt}
		\footnotesize
		
%import tableofcontents
% \input{./tableofcontents.tex}

\section{Allgemeines}
	\subsection{Primitive Datentypen}
		\begin{tabular}{p{.7cm} | p{2.1cm} | l}
			boolean&Boolescher Wert&true, false\\
			char&Textzeichen&'a','B','0','ë',...\\
			byte&Ganzzahl 8 Bit&-128 - 127\\
			short&Ganzzahl 16 Bit&-32'768 - 32'767\\
			int&Ganzzahl 32 Bit&-$2^{31}$ - $2^{31}$-1\\
			long&Ganzzahl 64 Bit&-$2^{63}$ - $2^{63}$-1, 1L\\
			float&Kommazahl 32 Bit&0.1f, 2e4f\\
			double&Kommazahl 64 Bit&0.1, 2e4\\
		\end{tabular}
		\subsubsection{Specials}
		a[double – int] $\rightarrow$ cannot convert double to int\\
		0/0$\rightarrow$ArithmeticException div by zero\\
		3.0/0 $\rightarrow$ Double.POSITIVE\_INFINITY\\
		f + '''' != f$\rightarrow$ f ist null\\
		g+1== g+2$\rightarrow$g=Double.POSITIVE\_INFINITY\\
		h != h$\rightarrow$h = Float/Double.NaN\\
		1 \% 0$\rightarrow$ArithmeticException div by zero\\
		0 \% 1$\rightarrow$0\\
		int<long<float$\rightarrow$Compiler-Fehler\\
		0.0 / 0.f$\rightarrow$NaN (Double.NaN)\\
		x + 1 == x $\rightarrow$ Double.POSITIVE\_INFINITY\\
		d++ == --d$\rightarrow$ true\\ 
		float vergleichen$\rightarrow$Math.abs(x – y) < delta\\
		\vspace{8pt}
	\subsection{Overloading \& Overriding}
	\begin{enumerate}
		\item Statischer Typ (links) anschauen
		\item (null) anschauen $\rightarrow$ spezifischster wählen
		\item Overriding beachten
	\end{enumerate}
	\textbf{Overloading:} Gleiche Methode wird mit anderem Typ überschrieben.\\
	\textbf{Overriding:} Gleiche Methode wird 1:1 überschrieben. (@Override)
	
	



	\subsection{Exceptions}
	\begin{lstlisting}
String method(String s) throws Exception {
if (s == null) {
throw new Exception("String is null"); } 
		\end{lstlisting}
		
		\subsubsection{Exceptions behandeln}
		\begin{lstlisting}
try {
... regular code ... } catch (Exception e) {
... error handling ... } finally {
	... statements ... }
		\end{lstlisting}
		\subsubsection{Eigene Exceptions}
		\begin{lstlisting}
class StringClipException extends Exception {
	public StringClipException() {}
	public StringClipException(String message) {
		super(message); }}
//------------------in Methode:------------------
String clip(String s) throws StringClipException {
	if (s == null) {
	throw new StringClipException("no string"); } }
		\end{lstlisting}
		\subsubsection{Checked und Unchecked Exceptions}
		\textcolor{b}{Checked Exceptions:}
		\begin{itemize}
			\item Exception muss behandelt werden (catch)
			\item throws-Deklaration im Methodenkopf
			\item Wird vom Compiler geprüft
			\item \tiny \texttt{ClassNotFoundExc., IllegalAccessExc., EigeneExc.}
		\end{itemize}
		\textcolor{b}{Unchecked Exceptions}
		\begin{itemize}
			\item Keine throws-Deklaration/Behandlung nötig
			\item Wird nicht vom Compiler geprüft
			\item \tiny \texttt{RuntimeExc., NullPointerExc., IndexOutOfBoundsExc., Error}
		\end{itemize}
		
	
\section{Vererbung}
	\subsubsection{Abstract Class}
	\begin{lstlisting}
public abstract class Vehicle {
	//Vererbbare Methoden....
	public abstract void drive();
	\end{lstlisting}
	\subsubsection{Erbende Klassen von Klassen}
	\begin{lstlisting}
public class Car extends Vehicle {
	@Override
	public void drive() {
	System.out.println("Car");
	//Ausfuehren der Vatermethode wenn nicht abstract
	super.drive; 
	\end{lstlisting}
	\subsubsection{Erben von Interfaces}
	\begin{lstlisting}
public class RegularCar implements Vehicle {...}
	\end{lstlisting}


	\columnbreak




\section{Interfaces}
	\subsection{Allgemeines}
	\begin{itemize}
		\item Facette einer Klasse
		\item Trennung Spezifikation und Implementation
		\item Lollipop oder Hierarchische Mehrfachvererbung
		\item Nur Klassen sind instanzierbar
		\item Schnittstellen sind aber als statische Typen verwendbar (linke Seite)
		\item Variablen in Interfaces sind Konstanten!
		\item Keine Instanzvariablen im Interface
	\end{itemize}
	\subsection{Vererbung}
		\subsubsection{Interface $\rightarrow$ Interface}
			\begin{lstlisting}
public interface Vehicle{...}
public interface LandMobile extends Vehicle{...}
			\end{lstlisting}
		\subsubsection{Interface $\rightarrow$ Klasse}
			\begin{lstlisting}
public interface Vehicle{...}
public class Car implements Vehicle{...}
			\end{lstlisting}
	\subsection{Default Methods}
	\begin{itemize}
		\item Falls Kollisionen bei Mehrfachvererbung 
		\item Ähnlich zu Mehrfachvererbung von abstrakten Klassen
		\item Keine Instanzvariablen in Schnittstelle!
		\item Methoden können in Klasse überschrieben werden
		\item Falls nicht überschrieben: Defaultmethode
	\end{itemize}
		\begin{lstlisting}
interface Vehicle {
	public default void printModel() {...method...}
		\end{lstlisting}
	\subsection{Functional Interface}


		\begin{tabular}{|p{1.6cm} p{3.1cm}|}
			\hline
			Lambda & @FunctionalInterface\\
			\hline
			\hline
			\tiny \texttt{x $\rightarrow$!x }&\tiny \texttt{interface BoolFunction \{ boolean apply(boolean x);\}}\\
			\hline
			\tiny \texttt{(x, y)$\rightarrow$x + y}&\tiny \texttt{interface IntFunction \{}\\
			&\tiny \texttt{int apply(int x, int y);\}}\\
			\hline
			\tiny \texttt{p ->p.getName() + p.getAge()}& \tiny \texttt{interface Function \{String apply(Person p);\}}\\
			\hline
			\tiny \texttt{()$\rightarrow$System.out. print(''OK'')}&\tiny \texttt{interface Runable \{void run();\}}\\
			\hline
			\tiny \texttt{(p1, p2) -> p1.compareTo(p2)}&\tiny \texttt{interface Comparable<T> \{ int compareTo(T other);\}}\\
			\hline
			\tiny \texttt{(x, y, z) -> x.compareTo(y)== y.compareTo(x)}&\tiny \texttt{interface TreeSome<T> \{ boolean apply(T first, T second, T third)\}}\\
			\hline
		\end{tabular}








\section{Lambda}

	\begin{lstlisting}
class Utils {
static <T> void removeAll(Collection<T> collection,
Predicate<T> criterion) {
var it = collection.iterator(); while (it.hasNext()) 
{if (criterion.test(it.next())) {it.remove(); }}}}
	
Utils.removeAll(people, p -> p.getAge() < 18); 
//Lokale Variabel muss bei Lambda zugriff final!
	\end{lstlisting}








\section{Generics}

	\subsubsection{Generische Klassen}
	\begin{lstlisting}
public class MyClass<T> {
...
public T add(T value1, T value2) { ... }
//----------------in Main Methode----------------
var intMyClass = new MyClass<Integer>(23);
	\end{lstlisting}
	\subsubsection{Generische Interfaces}
		\begin{lstlisting}
interface Iterator<E> {
	boolean hasNext();
	E next();}
		\end{lstlisting}
	\subsubsection{Generische Methoden}
	\begin{lstlisting}
public <E> Stack<E> multiPush(E value, int times) { 
	var result = new Stack<E>();
	for (int i = 0; i < times; i++) { 
		result.push(value); } }
	\end{lstlisting}
	\subsubsection{Type Bounds}
	\tiny \texttt{item.draw()} \footnotesize ist eine spezifische Methode, die nur von Grafik geerbten Klassen ausgeführt werden kann. Deshalb muss Type Bounds verwendet werden:
	\begin{lstlisting}
	class GraphicStack<T extends Graphic> {...}
	\end{lstlisting}



	\columnbreak



	\section{Collections}
	\subsection{List}
		\subsubsection{ArrayList}
			\begin{itemize}
				\item Duplikate möglich
				\item null ist auch einfügbar
				\item Add/Remove während Iterierens $\rightarrow$ Fehler
				\item get(), set(), add(): sehr schnell
				\item remove(), contains(): langsam
			\end{itemize}
			\begin{lstlisting}
ArrayList<String> stringList = new ArrayList<>();
	stringList.add("OO"); //Add Element
	stringList.add(0, "Bsys1"); //Add @ pos 0
	String x = stringList.get(1); //get @ pos 1
	stringList.set(0, "Bsys2"); //replace at pos 0
	boolean b = stringList.contains("CN1")
	stringList.remove("ICTh");
	stringList.remove(1); //remove @ pos 1
			\end{lstlisting}

		\subsubsection{LinkedList}
		\begin{itemize}
			\item Verkettete Liste der Elemente
			\item Duplikate möglich
			\item Dynamisch hinzufügbar und entfernbar
			\item Intern Doppelt verkettet
			\item add(), remove(anfang/ende) sehr schnell
			\item get(), set(), contains(), remove() langsam
		\end{itemize}
		\begin{lstlisting}
List<String> firstList = new ArrayList<>();
	//Gleiche Funktionen wie ArrayList
		\end{lstlisting}

		\subsubsection{Queue / Deques}
		\begin{itemize}
			\item Verkettete Liste der Elemente
			\item Dynamisch hinzufügbar und entfernbar
			\item LIFO oder FIFO möglich
			\item Intern Doppelt verkettet
		\end{itemize}
		\begin{lstlisting}
Deque<String> queue = new LinkedList<>();
	queue.addLast(elem);
	queue,addFirst(elem);
	queue.removeFirst(elem);
	queue.removeLast(elem);
		\end{lstlisting}

	\subsection{Set}
		\subsubsection{HashSet}
		\begin{itemize}
			\item Keine Duplikate
			\item Unsortiert 
			\item Oft sehr effizient
			\item In Hashtabelle gespeichert
			\item Elemente geben hashCode() konsistent zu equals()
		\end{itemize}
		\begin{lstlisting}
Set<String> otherSet = new HashSet<>();
	set.add(elem);
	set.addAll(list); //Collection of elem
	set.size();
	set.remove(elem);
	set.isEmpty();
	String[] a = (String[]) set.toArray();
		\end{lstlisting}
		\subsubsection{TreeSet}
		\begin{itemize}
			\item Keine Duplikate
			\item Sortiert
			\item immer effizient
			\item In Binärbäumen gespeichert
			\item Elemente implementieren Comparable und equals()
		\end{itemize}
		\begin{lstlisting}
Set<String> firstSet = new TreeSet<>();
	//Gleiche Funktionen wie TreeSet
		\end{lstlisting}
	
	\subsection{Map}
		\subsubsection{TreeMap}
		\begin{itemize}
			\item Mengen von Schlüssel-Wert-Paaren
			\item Nach Schlüssel sortiert
			\item immer effizient
		\end{itemize}
		\begin{lstlisting}
Map<Integer, String> bachelors = new TreeMap<>();
	map.put(20000, "Hello");
	map.containsKey(key);
	map.containsValue(value);
	String x = map.get(20000);
	for (int number : map.keySet()) { 
		System.out.println(number);}// print all elem
		\end{lstlisting}
		\subsubsection{HashMap}
		\begin{itemize}
			\item Mengen von Schlüssel-Wert-Paaren
			\item Unsortiert
			\item oft sehr effizient
		\end{itemize}
		\begin{lstlisting}
Map<Integer, String> map = new HashMap<>();
	//Gleiche Funktionen wie HashMap
		\end{lstlisting}



\columnbreak




\section{Stream API}
	Deklarative Abfragen von Collections.
	\begin{lstlisting}
people.stream().sorted().forEach(System.out::println);
	\end{lstlisting}
	\subsection{Zwischenoperationen}
		\begin{itemize}
			\item \tiny \textcolor{b}{\texttt{filter(Predicate)}}
				\footnotesize Rauspicken gem. Predicate
			\item \tiny \textcolor{b}{\texttt{map(Function)}}
				\footnotesize Projizieren gemäss Function
			\item \tiny \textcolor{b}{\texttt{mapToInt(Function)}}
				\footnotesize Proji. auf primitiver Typ
			\item \tiny \textcolor{b}{\texttt{sorted()}}
				\footnotesize Sortieren
			\item \tiny \textcolor{b}{\texttt{distinct}}
				\footnotesize Duplikate werden gelöscht
			\item \tiny \textcolor{b}{\texttt{limit(long n)}}
				\footnotesize Erste n Elemente liefern
			\item \tiny \textcolor{b}{\texttt{skip(long n)}}
				\footnotesize Erste n Elemente ignorieren
		\end{itemize}
	\subsection{Terminaloperationen}
		\begin{itemize}
			\item \tiny \textcolor{b}{\texttt{forEach(Consumer)}}
				\footnotesize Pro Element anwenden
			\item \tiny \textcolor{b}{\texttt{forEachOrderd(consumer)}}
				\footnotesize Erhält die Reihenfolge der Elemente
			\item \tiny \textcolor{b}{\texttt{count()}}
				\footnotesize Anzahl Elemente
			\item \tiny \textcolor{b}{\texttt{min(), max()}}
				\footnotesize Mit Comparator Argument
			\item \tiny \textcolor{b}{\texttt{average(), sum()}}
				\footnotesize Nur bei int, long, double
			\item \tiny \textcolor{b}{\texttt{findAny()}}
				\footnotesize Gibt irgendein Element zurück
			\item \tiny \textcolor{b}{\texttt{findFirst()}}
				\footnotesize gibt erstes Element zurück
		\end{itemize}
	\subsection{Rückumwandlung}
		\textcolor{b}{Zu Array:}
		\begin{lstlisting}
Person[] array = peopleStream.toArray(Person[]::new)
		\end{lstlisting}
		\textcolor{b}{Zu Collection:}
		\begin{lstlisting}
List<Person> list = peopleStream.collect(Collectors.toList());
		\end{lstlisting}	
		\subsubsection{Gruppieren}
			\begin{lstlisting}
Map<Integer, List<Person>> peopleByAge = 
people.stream()
.collect(Collectors.groupingBy(Person::getAge));
			\end{lstlisting}




\section{Input \& Output}
	\subsection{Byte-Streams}
		\subsubsection{Input}
			\begin{lstlisting}
var in = new FileInputStream("myFile.data"); 
int value = in.read(); 
while (value >= 0) {  //-1 = end
	byte b = (byte)value;
	// work with b
	value = in.read(); }
in.close();
			\end{lstlisting}
		

		\subsubsection{Output}
			\begin{lstlisting}
var out = new FileOutputStream("test.txt"); 
while (...) {
	byte b = ...;
	out.write(b); }
out.close(); 
			\end{lstlisting}

	\subsection{Character-Strams}
		\subsubsection{Zeichen lesen mit File Reader}
			\begin{lstlisting}
try (var reader = new FileReader("quotes.txt", StandardCharsets.UTF_8)){ 
	int value = reader.read();
	while (value >= 0) {
		char c = (char)value;
		// use character
		value = reader.read();}}
			\end{lstlisting}
			\subsubsection{Zeilenweise Lesen mit Buffered Reader}
			\begin{lstlisting}
try (var reader = new BufferedReader(new FileReader("quotes.txt")) {
	String line = reader.readLine(); 
	while (line != null) {
		System.out.println(line);
		line = reader.readLine();}}
			\end{lstlisting}
			\subsubsection{Output}
				\begin{lstlisting}
try (var writer = new FileWriter("test.txt", StandardCharsets.UTF_8, true)){
	writer.write("Hi");	}
				\end{lstlisting} 

		\subsection{Text-Datei-Zugriff}
			\subsubsection{Ganze Text-Datei einlesen}
				\begin{lstlisting}
List<String> lines = Files.readAllLines(Path.of("in.txt"), StandardCharsets.UTF_8);
				\end{lstlisting}
			\subsubsection{Alle Zeilen als Stream API}
				\begin{lstlisting}
Stream<String> lines = Files.lines(Path.of("in.txt"), StandardCharsets.UTF_8);
				\end{lstlisting}
			\subsubsection{Ganze Text-Datei schreiben}
				\begin{lstlisting}
Files.write(Path.of("out.txt"), lines, StandardCharsets.UTF_8);
				\end{lstlisting}

		\subsection{Scanner}
		\begin{lstlisting}
public static String readConsole(String text) {
	Scanner scanner = new Scanner(System.in);
	String input = scanner.next();
	scanner.close(); return input; }
		\end{lstlisting}


\section{Unit Testing}
\begin{lstlisting}
class StackTest {
@Test 
void testEmptyStack() {
	assertArrayEquals(new int[]{2,4,6}, //erwartet
	aufgerufeneMethode(new int[]{1,2,3}))} //eingabe
	\end{lstlisting}
	\begin{tabular}{|p{2cm} p{2.65cm}|}
		\hline
		Test&Bedingung\\
		\hline
		\hline
		\tiny\texttt{assertEquals (expected, actual)}&\footnotesize Objects.equals(actual, expected) für primitive- und Referenz-Typen\\
		\hline
		\tiny\texttt{assertNotEquals (expected, actual)}&\footnotesize !Objects.equals(actual, expected)\\
		\hline
		\tiny\texttt{assertSame (expected, actual)}&\footnotesize actual == expected\newline $\rightarrow$ Referenzvergleich!!\\
		\hline
		\tiny\texttt{assertNotSame (expected, actual)}&\footnotesize expected != actual (Referenzvergleich) (nur für Referenztypen verwenden!)\\
		\hline
		\tiny\texttt{assertTrue(condition)}&\footnotesize condition\\
		\hline
		\tiny\texttt{assertFalse(condition)}&\footnotesize !condition\\
		\hline
		\tiny\texttt{assertNull(value)}&\footnotesize value == null\\
		\hline
		\tiny\texttt{assertNotNull(value)}&\footnotesize value != null\\
		\hline
		\tiny\texttt{fail(description)}&\footnotesize Immer verletzt\\
		\hline
		\tiny\texttt{assertThrows (StackException.class, () -> stack.pop())}&\footnotesize Exception\\
		\hline
	\end{tabular}

\section{Regex}

\begin{center}
	\begin{tabular}{| p{1.8cm} p{.8cm} p{1.6cm}|}
		\hline
		&Regex&Gültige Texte\\
		\hline
		\hline
		Konstante&Mi&''Mi''\\
		\hline
		Alternative&$Mo\textcolor{red}{|}Di$&''Mo'', ''Di''\\
		\hline
		Gruppierung&\textcolor{red}{(}M$|$D\textcolor{red}{)(}i$|$o\textcolor{red}{)}&''Mo'', ''Di'',..\\
		\hline
		Option&-\textcolor{red}{?}1&''-1'', ''1''\\
		\hline
		0 bis beliebig&1\textcolor{red}{*}&'''', ''1'', ''11'', ''111'', usw.\\
		\hline
		1 bis beliebig&1\textcolor{red}{+}&''1'',''11",''111''\\
		\hline
		min bis max&1\textcolor{red}{\{2,4\}}&''11'', ''111'',...\\
		\hline
		min bis beliebig&1\textcolor{red}{\{2,\}}&''11'',''111'',...\\
		\hline
		exakte Anzahl&1\textcolor{red}{\{3\}}&''111''\\
		\hline
		Aufzählung&\textcolor{red}{[}aeiou\textcolor{red}{]}&''a'', ''e'', ''i'',\\
		\hline
		Bereich&\textcolor{red}{[}a\textcolor{red}{-}z\textcolor{red}{]}&''a'', bis ''z''\\
		\hline
		Ausschluss&\textcolor{red}{[\textasciicircum}a\textcolor{red}{]}&Jedes Zeichen ausser ''a''\\
		\hline
		Vereinigung&\textcolor{red}{[}a\textcolor{red}{-}zA\textcolor{red}{-}Z\textcolor{red}{]}&a-z ||A-Z\\
		\hline
		Whitespace&\textcolor{red}{$\backslash$s}&Whitespace\\
		\hline
		Joker&\textcolor{red}{.}&Jedes Zeichen ausser $\backslash$n\\
		\hline
		Escapen&\textcolor{red}{$\backslash$$\backslash$}?&?\\
		\hline
	\end{tabular}
\end{center}

\vspace{-10pt}

	%\columnbreak




\section{Beispiele}
	\subsubsection{Zyklen}
		\begin{lstlisting}
void checkAcylic(Atom current) {
	checkAcylic(current, new HashSet<>());}
			
void checkAcylic(Atom current, Set<Atom> visited) {
	if (visited.contains(current)) {
		throw new RuntimeException("Cyclic!"); }
	visited.add(current);
	for (Atom follower : current.getZerfall()) {
		checkAcylic(follower, new HashSet<>(visited));}}
		\end{lstlisting}
	\subsubsection{Array umkopieren}
	\begin{lstlisting}
void resort(int[] array) {
	int index = unsortedIndex(array); //Index
	int value = array[index]; 
	delete(array, index); 
	insert(array, value); }

void delete(int[] array, int position) {
	for (int i = position; i < array.length-1; i++) {
	array[i] = array[i + 1]; 
	array[array.length - 1] = Integer.MAX_VALUE;}}

void insert(int[] array, int value) { 
	int position = 0;
	while (position < array.length && value > array[position]) { position++; }
	for (int i = position; i < array.length-1; i++) {
	array[i + 1] = array[i]; }array[position] = value;}
	\end{lstlisting}
	\subsubsection{Max-Clique}
		\begin{lstlisting}
Set<Person> maxClique() {
	Set<Person> result = new HashSet<>();
	for (Set<Person> group : allSubsets( new ArrayList<>(allPeople))) {
	if (isClique(group)&&result.size()<group.size()) {
			result = group;} } return result; }

boolean isClique(Set<Person> group) { 
	for (Person first : group) {
		for (Person second : group) {
			if (first != second && 
			   !first.getFriends().contains(second)) {
				return false; } } } return true; }

Set<Set<Person>> allSubsets(List<Person> people) { 
	Set<Set<Person>> result = new HashSet<>(); 
	result.add(new HashSet<>());
	for (Person current : people) {
		List<Person>remaining = new ArrayList<>(people); 
		remaining.remove(current);
		for (Set<Person>subset : allSubsets(remaining)){
			  result.add(subset);
			  Set<Person>augmented =new HashSet<>(subset); 
			  augmented.add(current); 
			  result.add(augmented);} } return result; }

		\end{lstlisting}
	\subsubsection{Main Methode}
		\begin{lstlisting}
class Main {
	public static void main(String[] args) {...} }		
		\end{lstlisting}
	\subsubsection{Compare To}
		\begin{lstlisting}
@FunctionalInterface
interface Comparable<T> {
	int compareTo(T other); }
		
class Person implements Comparable<Person> {
	private int age;
	@Override 
	public int compareTo(Person other) {
		return Integer.compare(age, other.age); }
//oder
	@Override 
	public int compareTo(Person other) {
		int result = lastName.compareTo(other.lastName);
		if (result == 0) {
			result = firstName.compareTo(other.firstName); } 
		return result;}}
//oder
	people.sort(this::compareByAge);//in Methode
//oder
	(p1, p2) -> p1.compareTo(p2)
		\end{lstlisting}
	\subsubsection{Alle indirekten und direkten Elemente}
		\begin{lstlisting}
public Set<Atom> getAllSubAtoms(Atom current) {
	Set<Atom> result = new HashSet<>();
	for (Atom directAtom : current.zerfallsElemente) {
		result.add(directAtom);
		for (Atom indirectAtom : getAllSubAtoms(directAtom)) {
		result.add(indirectAtom);}}
			return result;}
		\end{lstlisting}
	\subsubsection{Alle indirekten und direkten Elemente zählen}
		\begin{lstlisting}
int count(List<Chapter> listOfContents) { 
	int total = listOfContents.size();
	for (Chapter chapter : listOfContents) {
		total += count(chapter.getSubChapters()); }
		return total; }
		\end{lstlisting}
	\subsubsection{Alle Wörter aus Buchstaben generieren}
		\begin{lstlisting}
Set<String> generateWords(Set<Character> input) { 
	Set<String> result = new HashSet<>();
	if (input.size() == 0) { result.add(""); }
	for (Character letter : input) {
		Set<Character> remainder = new HashSet<>(input); 
		remainder.remove(letter);
		for (String word : generateWords(remainder)) {
			result.add(word + letter);}}
	 return result;}
		\end{lstlisting}
	\subsubsection{Read \& Write @ same time}
		\begin{lstlisting}
void meth(String in, String out) throws IOException{ 
	try (FileReader reader = new FileReader(in);
			 FileWriter writer = new FileWriter(out)) { 
			 int value = reader.read();
			 while (value >= 0) {
			  char character = (char)value; 
			  	//Work with character....
				writer.write(character));
			 	value = reader.read(); } } }
		\end{lstlisting}
	\subsubsection{toSting()}
		\begin{lstlisting}
//Vaterklasse:
public abstract String toString(int indent);			

@Override
public String toString() {
	return toString(0);}

protected String getIndent(int length) { 
	String text = ""; } }
  for (int i = 0; i < length; i++) { text += " ";}
  return text;

//Erbende Klasse:
@Override
public String toString(int indent) {
	String text = getIndent(indent) + getName() +"\n";
	for (Food food : ingredients) {
	text += food.toString(indent + 1);}
return text; } }
		\end{lstlisting}
	\subsubsection{Substring / Split}
		\begin{lstlisting}
int number = Integer.parseInt(line.substring(0, 3));
String name = line.substring(4);

String a = "1234 10"; String[] c = a.split(" ");
		\end{lstlisting}

	\subsubsection{Optimum an Elementen finden}	
		\begin{lstlisting}
Set<Item> solveRucksack(List<Item> allItems, 
int maxWeight) { 
	Set<Set<Item>>allCombinations =powerSet(allItems); 
	Set<Item> optimum = new HashSet<>();
	for (Set<Item> combination: allCombinations) {
		if (totalWeight(combination) <= maxWeight && 
			totalValue(combination)>totalValue(optimum)){
				optimum = combination; } } 
	return optimum; }

Set<Item> solve(List<Item> allItems, int maxWeight){ 
	Set<Set<Item>>allCombinations =powerSet(allItems); 
	Set<Item> optimum = new HashSet<>(); //result
	for (Set<Item> combination: allCombinations) {
			if (totalWeight(combination) <= maxWeight && 
			totalValue(combination) >totalValue(optimum)){
					optimum = combination; }} return optimum;}

int totalValue(Set<Item> combination) {
	return combination.stream().mapToInt(item -> item.getValue()).sum();}

int totalWeight(Set<Item> combination) {
	return combination.stream().mapToInt(item -> item.getWeight()).sum();}
		\end{lstlisting}
	\subsubsection{powerSet (alle Kombinationen)}
		\begin{lstlisting}
<T> Set<Set<T>> powerSet(List<T> elements) {
	if (elements.size() == 0) {
		Set<Set<T>> empty = new HashSet<>(); 
		empty.add(new HashSet<>()); return empty; } }

	List<T> remaining = new ArrayList<>(elements);
	T current = remaining.remove(0);
	Set<Set<T>> result = new HashSet<>();
	for (Set<T> set: powerSet(remaining)) {
		result.add(set);
		Set<T> augmentedSet = new HashSet<>(set); 
		augmentedSet.add(current); 
		result.add(augmentedSet); }
	return result;
		\end{lstlisting}
	
	
	\subsubsection{MagicSquare Test}
		\begin{lstlisting}
public boolean isMagicSquare(int[][] square) { 
	int sum1 = 0, sum2 = 0;
    for(int i = 0; i < square.length; i++) { 
		sum1 += square[i][i];
		sum2 += square[i][square.length-1-i];
		return sum1 == sum2; } }
		\end{lstlisting}
	\subsubsection{MagicSquare Generator}
		\begin{lstlisting}
int[][] computeMagicSquare(int size){
	var square = new int[size][size];
	if (fill(square, 0)) { return square;
	} else { return null; //no solution } }

boolean fill(int[][] square, int position) {
	int size = square.length;
	if (position == size * size) {
		return isMagicSquare(square); }
	for (int i = 1; i <= size * size; i++) {
		if (available(sqare, position, number)) {
			if (fill(square, position + 1)){
				return true; } } }
	return false; }

boolean available(int[][] square, int position, int number) {
	int size = square.length;
	for (int index = 0; index < position; index++){
		if (square[index/size][index%size] == number) {
			return false; } }
	return true; }
		\end{lstlisting}
	
	\subsubsection{Alle Binärbäume}
	\begin{lstlisting}
class Node {
	private final Node left, right; 
	// Konstruktor und Getter }

Set<Node> generate(int amount) {
	var result = new HashSet<Node>(); 
	if (amount == 0) { result.add(null); }
	for (int part = 0; part < amount; part++) {
		for (Node left : generate(part)) {
			for (Node right : generate(amount-part-1)) {
				result.add(new Node(left, right)); } } }
		return result;}
	\end{lstlisting}
	\subsubsection{Fakultät}
		\begin{lstlisting}
staic long fact(long i){
	if (i==0) return 1; else return i * fac(i - 1);}
		\end{lstlisting}

	\subsubsection{Switch - Case}
		\begin{lstlisting}
int x = 5; switch(x){
	case 2: Syso("2"); break;
	case 5: Syso("5"); break;
	dafault: Syso("No match!"); break; }
		\end{lstlisting}
	\subsubsection{Lambda Beispiele}
	Syntax: (Parameterliste) -> {Body}
	\begin{lstlisting}

stream().sort((p1, p2) -> 
	Integer.compare(p1.getAge(), p2.getAge()))

stream().filter(p -> p.getAge() >= 18)

stream().filter( p -> p instanceof House)

stream().filter( p -> p.getGender
				.equals(Gender.FEMALE))

stream().map(p -> p.getName()) //to String

//Top 10 earner
people.stream()
	.sorted(Comparator.comparing(
		Person::getSalary).reversed())
	.mapToInt(p -> p.getSalary())
	.limit(10)
//get Max Age from People in Zurich
people.stream()
	.filter(p -> p.getCity().equals("Zuerich"))
	.mapToInt(p -> p.getAge())
	.max()
	.getAsInt();
//get Average Age of Male
var x = people.stream()
	.filter((p) -> p.getGender().equals(Gender.MALE))
	.mapToInt(p -> p.getAge())
	.average();
//Get all female Names with 3 or less Characters
	people.stream()
	.filter((p) -> p.getGender().equals(Gender.FEMALE) 
		&& p.getFirstName().length() <= 3)
	.map(p -> p.getFirstName())
//Durchschnittsalter pro Ort (mittels Collector)
people.stream()
	.collect(Collectors.groupingBy(Person::getCity, 
			Collectors.averagingInt(Person::getAge)))
	.forEach((city, age) -> 
			System.out.println(city + " " + age));
//All Elements in Object cotained in List
catalogue.allProperties()
	.filter(estate -> estate instanceof House)
	.flatMap((estate -> 
			((House)estate).getParkings().stream()));
	\end{lstlisting}

	\vspace{30pt}

	\subsubsection{Stream Beispiele}
		\begin{lstlisting}
people.stream().map(p->p.name)
IntStream.range(1, 100) // Zahlen von bis 100
Stream.of(2,3,5,7) // Eigene Aufzaehlung
Stream.concat(stream1, stream2) // 1 gefolgt von 2
		\end{lstlisting}


	
	\subsection{compareTo}
	\subsubsection{Comparator}
		\begin{lstlisting}
class AgeComparator implements Comparator<Person> {
	@Override
	public int compare(Person first, Person second){
	return Integer.compare(first.getAge(), second.getAge() ); }
		\end{lstlisting}
	\subsubsection{Comparable}
		\begin{lstlisting}
class Person implements Comparable<Person> {
	//.....
	@Override
	public int compareTo(Person other){
		int result = lastName.compareTo(other.lastName);
		if (result == 0){ 
		 result = firstName.compareTo(other.firstName);}
		return result;}
		\end{lstlisting}
	\subsection{Diverses Rekursives}
	\subsubsection{Palindrom}
	\begin{lstlisting}
printPalindroms("", "", 0, 5); //in Main

void printPalindroms(String prefix, String suffix, 
int pos, int length) {
	if (pos == (length + 1) / 2) {
		System.out.println(prefix + suffix);
	} else {
		boolean middle = position == length / 2;
		for (char c = 'A'; c <= 'Z'; c++) {
			printPalindroms(prefix + c, middle ? 
			suffix : c + suffix, position + 1, length);}}}
	\end{lstlisting}
	\subsubsection{Fibonacci}
		\begin{lstlisting}
static long fib(long n) {
	if (n == 0) {return 0;
	} else if(n == 1) { return 1;
	} else { 
		return fib(n-1) + fib(n-2);	} }
		\end{lstlisting}
	\subsubsection{Cluster von Personen}
		\begin{lstlisting}
boolean isLinked(Person from, Person to){
	return isLinked(from, to, new HashSet<>()); }

boolean isLinked (Person from, Person to, 
Set<Person> visited) {
	if (from == true) { return true;}
	if (visited.contains(from)){
		return false; }//dont check twice
	visited.add(from);
	for ( Person other: from.knownPeople()){
		if (isLinked(other, to, visited)){
			return true	} }
	return false; }
		\end{lstlisting}

	\subsection{Sort}
	\subsubsection{Bubble Sort}
	\begin{lstlisting}
boolean swapping;
do{	swapping = false;
	for(int i=0; i<array.length-1; i++){
		if(array[i] > array[i+1]){
			swap(array, i, i+1);
			swapping = true; } }
		} while(swapping);
	\end{lstlisting}

	\subsubsection{Quicksort}
		\begin{lstlisting}
void quicksort(int[] array, int left, int right){
	int pivot = array[(left+right)/2];
	int i = left, j = right;
	while(i <=j ){
		while(array[i] < pivot){i++; }
		while(array[j] > pivot){j--; }
		if(i <= j){
		swap(array, i, j); i++; j--; } }
	if(left < j){quicksort(array, left, j); }
	if(i < right){quicksort(array, i, right); } }
		\end{lstlisting}



	\subsection{NP-vollständig}
	\subsubsection{HAMPATH}
		\begin{lstlisting}
List<City> shortestRoundtrip(List<City> visited) { 
	if (visited.size() == allCities.size()) {
		return visited; }
	int min = Integer.MAX_VALUE; 
	List<City> best = null;
	for (City next : allCities) {
		if (!visited.contains(next)) {
			List<City>expanded = new ArrayList<>(visited); 
			expanded.add(next);
			List<City>result =shortestRoundtrip(expanded); 
			int distance = totalDistance(result);
			if (distance < min) {
				min = distance; 
				best = result; } } } return best; }

int totalDistance(List<City> roundtrip) { 
	int sum = 0;
	for (int index=0;index<roundtrip.size();index++){
		City current = roundtrip.get(index);
		City next = roundtrip.get((index + 1) % roundtrip.size()); 
		sum += current.getConnections().get(next);
		} return sum; }

		\end{lstlisting}









	


\end{multicols*}

% \input{./appendix.tex}

\end{document}